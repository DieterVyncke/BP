 \section{Definitie van een CMS}
CMS of Content Management Syste(e)m is een term die verschillende definities kan aannemen afhankelijk van de indruk van een persoon of project. CMS kan ook benamingen aannemen als ECM (Enterprise Content Management) of WCM (Web Content Management). Dit zijn alle benoemingen die bedrijven in het leven hebben geroepen om zich te onderscheiden.
\newline\newline
Als we het binnen de web ontwikkeling hebben over een Content Management Systeem, spreken we over een CMS. Een CMS biedt de mogelijkheid om met verschillende gebruikers die al dan niet verschillende toegangsrechten hebben, stukken inhoud van een website aan te passen zonder kennis van programmeren. Met inhoud beheren bedoelen we het toevoegen, aanpassen, archiveren en publiceren van informatie. Deze bewerkingen gebeuren via een controle paneel waar de verschillende rechten tot uiting komen. Afhankelijk van de rechten is het al dan niet mogelijk meer bewerkingen uit te voeren. Zo zal bijvoorbeeld een blogger enkel zijn eigen posten kunnen aanspreken, terwijl een beheerder (Admin) de posten van alle geregistreerde gebruikers zal kunnen beheren.
\newline\newline
Of een website al dan niet een CMS dient te hebben is afhankelijk van de klant. Het is zo dat de klant een stuk minder afhankelijk is van de web ontwikkelaar, maar is zeker ook niet onafhankelijk. Wanneer updates moeten doorgevoerd worden zal dit moeten gebeuren via de web ontwikkelaar. Aanpassingen buiten de afgebakende beheersbare gebieden, kan de klant niet zelf uitvoeren \citep{KohanBernard2010WhatCMS}.

 \section{Meest populaire CMS'en}
We bekijken de drie CMS'en die momenteel het grootste aandeel hebben. We lijsten de systemen op in volgorde van populariteit: WordPress, Joomla, Drupal \citep{BuiltWith2015CMSStatistics,W3Techs2016UsageWebsites}.
\newline\newline
Elk van deze systemen heeft als hoofddoel voor de eindgebruiker een vlot beheersbare site te creëren waarbij hij zelf de content kan beheren. Toch hebben deze systemen elk hun eigen specifieke doel.

\subsection{Wordpress}
WordPress is uitgegroeid tot de grootste globale blog applicatie. Dit systeem is een open source project, waar velen meewerken om dit product iedere dag te verbeteren. WordPress is een bijzonder eenvoudig en makkelijk zelf aan te leren systeem. WordPress werd in eerste instantie vooral gebruikt als blog site. Maar ondertussen is dit systeem verder uitgebreid met verschillende plugins, Widgets en thema's. Waardoor het ook mogelijk is om grotere sites te ontwikkelen. 

\subsection{Joomla}
Joomla is een stuk gecompliceerder en beter uitbreidbaar in vergelijking met WordPress, maar blijft toch eenvoudig in gebruik. Qua complexiteit en mogelijkheden ligt het in het midden. Uit deze selectie, is Joomla het meest geschikt voor het opzetten van sociale netwerk en e-commerce sites. Joomla combineert de voordelen van WordPress en Drupal en voegt hierbij zijn eigen features toe. 

\subsection{Drupal}
Drupal heeft een wijd aspect van mogelijkheden, het is daarom ook een meer geavanceerd systeem in vergelijking met de andere systemen. Drupal kan gebruikt worden voor het ontwikkelen van eenvoudige informatieve sites tot lmeer complexe trafiek intensieve sites voor grote bedrijven. Doordat Drupal vrij ingewikkeld is, schrikt dit de grootste groep mensen af die geen kennis van programmeren hebben. Het is aangewezen om een minimum kennis van HTML en PHP te hebben als je eraan denkt een website te bouwen via Drupal.